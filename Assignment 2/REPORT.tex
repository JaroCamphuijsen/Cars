\documentclass[a4paper]{article}
\usepackage[english]{babel}
\usepackage{pdfpages, titling}

\usepackage{array, float}
\usepackage{cite, amsmath, tablefootnote}
\usepackage{graphicx, caption, subfigure, wrapfig}
\usepackage{listings}
\usepackage{color}

\newcommand{\subtitle}[1]{%
  \posttitle{%
    \par\end{center}
    \begin{center}\large#1\end{center}
    \vskip0.5em}%
}
\lstset{frame=tb,
  language=Java,
  aboveskip=3mm,
  belowskip=3mm,
  showstringspaces=false,
  columns=flexible,
  basicstyle={\small\ttfamily},
  numbers=none,
  breaklines=true,
  breakatwhitespace=true,
  tabsize=3
}

\author{Eva van Weel() and Jaromir Camphuijsen (6042473)}
\date{\today}
\title{Mystery Dataset}
\subtitle{}
\begin{document}
\maketitle


\tableofcontents


\newpage
\section{Introduction}
Background, overview of relevant literature, structure of remainder of the report

Introduce the problem that you have studied
�Describe what others have done (related work)
�Describe the structure of your report
�Make sure that you include enough and appropriate references.
�Explain for each referenced paper why it is relevant for your paper.
�In some cases, depending on the size of each part, the problem statement and the related work should be made into separate sections or chapters.

\section{Core (multiple sections)}
Your report must clearly explain what it is about, and should not just be a repository of facts.
�The structure of this part depends on the nature of your work. E.g.:
�Start by describing your methods and algorithms
�Use formulae, pseudo-code
�Describe the implementation
�Describe import choices, leave out what should be self-evident
�But (well documented) source code goes to appendices, if to be included at all
�Describe your experiment and the results
�Interpret your results
�As with the introduction, the core may be split into separate sections, like methods, experimentation and interpretation of the results.

\section{conclusions}
In this section you give a clear resume of the conclusions from your work and discuss its relation to previous research.
�For that reason, some authors will actually put the section on related work after the core section(s)
�The conclusions should reflect back on the original problem statement in your introduction.
�When applicable, your work may result in suggestions on how to further validate or extend your research.


\bibliography{REPORTbib}
\bibliographystyle{ieeetr}
\end{document}